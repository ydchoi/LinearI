\documentclass{article} % For LaTeX2e
\usepackage{nips14submit_e,times}
\usepackage{amsmath}
\usepackage{amsthm}
\usepackage{amssymb}
\usepackage{mathtools}
\usepackage{hyperref}
\usepackage{url}
\usepackage{algorithm}
\usepackage[noend]{algpseudocode}
%\documentstyle[nips14submit_09,times,art10]{article} % For LaTeX 2.09

\usepackage{graphicx}
\usepackage{caption}
\usepackage{subcaption}

\def\eQb#1\eQe{\begin{eqnarray*}#1\end{eqnarray*}}
\def\eQnb#1\eQne{\begin{eqnarray}#1\end{eqnarray}}
\providecommand{\e}[1]{\ensuremath{\times 10^{#1}}}
\providecommand{\pb}[0]{\pagebreak}

\newcommand{\E}{\mathrm{E}}
\newcommand{\Var}{\mathrm{Var}}
\newcommand{\Cov}{\mathrm{Cov}}

\def\Qb#1\Qe{\begin{question}#1\end{question}}
\def\Sb#1\Se{\begin{solution}#1\end{solution}}

\newenvironment{claim}[1]{\par\noindent\underline{Claim:}\space#1}{}
\newtheoremstyle{quest}{\topsep}{\topsep}{}{}{\bfseries}{}{ }{\thmname{#1}\thmnote{ #3}.}
\theoremstyle{quest}
\newtheorem*{definition}{Definition}
\newtheorem*{theorem}{Theorem}
\newtheorem*{lemma}{Lemma}
\newtheorem*{question}{Question}
\newtheorem*{preposition}{Preposition}
\newtheorem*{exercise}{Exercise}
\newtheorem*{challengeproblem}{Challenge Problem}
\newtheorem*{solution}{Solution}
\newtheorem*{remark}{Remark}
\usepackage{verbatimbox}
\usepackage{listings}
\title{Linear Algebra I: \\
Problem Set I}


\author{
Youngduck Choi \\
CIMS \\
New York University\\
\texttt{yc1104@nyu.edu} \\
}


% The \author macro works with any number of authors. There are two commands
% used to separate the names and addresses of multiple authors: \And and \AND.
%
% Using \And between authors leaves it to \LaTeX{} to determine where to break
% the lines. Using \AND forces a linebreak at that point. So, if \LaTeX{}
% puts 3 of 4 authors names on the first line, and the last on the second
% line, try using \AND instead of \And before the third author name.

\newcommand{\fix}{\marginpar{FIX}}
\newcommand{\new}{\marginpar{NEW}}

\nipsfinalcopy % Uncomment for camera-ready version

\begin{document}


\maketitle

\begin{abstract}
This work contains the solutions to the problem set I
of Linear Algebra I 2015 at Courant Institute of Mathematical Sciences.
\end{abstract}

\bigskip

\begin{question}[1]
\end{question}
\begin{solution}
We show that the three vectors $u+v+w$, $v+w$, and $w$ are linearly independent.
Assume that 
\eQb
a_1(u+v+w) + a_2(v+w) + a_3(w) &=& 0.
\eQe
This is equivalent to 
\eQb
a_1u + (a_1+a_2)v + a_3(u+v+w) &=& 0.
\eQe
This is again equivalent to
\eQb
a_1u + a_2v + a_3w &=& 0.
\eQe
Since the set $\{ u,v,w \}$ are linearly independent, we have that $a_1 = 0$
Hence, $u+v$ 
$\qed$
\end{solution}

\bigskip

\begin{question}[2]
\end{question}
\begin{solution}

\end{solution}


\bigskip

\begin{question}[3]
\end{question}
\begin{solution}
\end{solution}

\bigskip

\begin{question}[4]
\end{question}
\begin{solution}
Let $W_1$ and $W_2$ be subspaces of $V$. First, assume that
$W_1 \subseteq W_2$. Then, we have $W_1 \cup W_2 = W_2$. Since
$W_2$ is a subspace, we have that $W_1 \cup W_2$ is a subspace. By symmetry, we also have
if $W_2 \subseteq W_1$, then $W_1 \cup W_2$ is a subspace.

\smallskip

Now, assume that $W_1 \cup W_2$ is a subspace. 
Suppose that $W_1 \setminus W_2 \neq \emptyset$. 
Let $x \in W_1$ and $y \in W_1 \setminus W_2$.


Hence, we have shown that $W_1 \cup W_2$ is a subspace iff $W_1 \subseteq W_2$
or $W_2 \subseteq W_1$.  

\end{solution}

\bigskip

\begin{question}[5]
\end{question}
\begin{solution}
\end{solution}

\bigskip

\begin{question}[6]
\end{question}
\begin{solution}
Assume that 
\end{solution}

\bigskip

\begin{question}[7]
\end{question}
\begin{solution}
\end{solution}

\bigskip

\begin{question}[8]
\end{question}
\begin{solution}
\end{solution}

\end{document}
