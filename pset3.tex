\documentclass{article} % For LaTeX2e
\usepackage{nips14submit_e,times}
\usepackage{amsmath}
\usepackage{amsthm}
\usepackage{amssymb}
\usepackage{mathtools}
\usepackage{hyperref}
\usepackage{url}
\usepackage{algorithm}
\usepackage[noend]{algpseudocode}
%\documentstyle[nips14submit_09,times,art10]{article} % For LaTeX 2.09

\usepackage{graphicx}
\usepackage{caption}
\usepackage{subcaption}

\def\eQb#1\eQe{\begin{eqnarray*}#1\end{eqnarray*}}
\def\eQnb#1\eQne{\begin{eqnarray}#1\end{eqnarray}}
\providecommand{\e}[1]{\ensuremath{\times 10^{#1}}}
\providecommand{\pb}[0]{\pagebreak}

\newcommand{\E}{\mathrm{E}}
\newcommand{\Var}{\mathrm{Var}}
\newcommand{\Cov}{\mathrm{Cov}}

\def\Qb#1\Qe{\begin{question}#1\end{question}}
\def\Sb#1\Se{\begin{solution}#1\end{solution}}

\newenvironment{claim}[1]{\par\noindent\underline{Claim:}\space#1}{}
\newtheoremstyle{quest}{\topsep}{\topsep}{}{}{\bfseries}{}{ }{\thmname{#1}\thmnote{ #3}.}
\theoremstyle{quest}
\newtheorem*{definition}{Definition}
\newtheorem*{theorem}{Theorem}
\newtheorem*{lemma}{Lemma}
\newtheorem*{question}{Question}
\newtheorem*{preposition}{Preposition}
\newtheorem*{exercise}{Exercise}
\newtheorem*{challengeproblem}{Challenge Problem}
\newtheorem*{solution}{Solution}
\newtheorem*{remark}{Remark}
\usepackage{verbatimbox}
\usepackage{listings}
\title{Linear Algebra I: \\
Problem Set II}


\author{
Youngduck Choi \\
CIMS \\
New York University\\
\texttt{yc1104@nyu.edu} \\
}


% The \author macro works with any number of authors. There are two commands
% used to separate the names and addresses of multiple authors: \And and \AND.
%
% Using \And between authors leaves it to \LaTeX{} to determine where to break
% the lines. Using \AND forces a linebreak at that point. So, if \LaTeX{}
% puts 3 of 4 authors names on the first line, and the last on the second
% line, try using \AND instead of \And before the third author name.

\newcommand{\fix}{\marginpar{FIX}}
\newcommand{\new}{\marginpar{NEW}}

\nipsfinalcopy % Uncomment for camera-ready version

\begin{document}


\maketitle

\begin{abstract}
This work contains the solutions to the problem set II
of Linear Algebra I 2015 at Courant Institute of Mathematical Sciences.
\end{abstract}

\bigskip

\begin{question}[1]
\end{question}
\begin{solution}
\end{solution}

\pagebreak

\begin{question}[2]
\end{question}
\begin{solution} Let $A$ and $B$ be square matrices. Assume that $AB$ is
invertible.  
\end{solution}

\newpage

\begin{question}[3]
\end{question}
\begin{solution}
Let $A$ be a square matrix.
From the matrix multiplication rule, we have
\eQb
(A A^T)_{ii} &=& \sum_{k} a_{ik}b_{ki}, \\
\eQe
where $a_{ik}$ denotes the $(i,k)$th entry of the matrix $A$ and
$b_{ki}$ denotes the $(k,i)$th entry of the matrix $A^T$. By 
definition of transpose, it follows that $a_{ik} = b_{ki}$, and 
we obtain
\eQb 
(A A^T)_{ii} &=& \sum_{k} a_{ik}^2.
\eQe
Therefore, by definition of trace, we have
\eQb
\text{tr}(AA^T) &=& \sum_{i,k} a_{ik}^2.
\eQe
Since $a_{ik}^2 \geq 0$ for all $i,k$, it follows that
\eQb
\text{tr}(AA^T) &\geq& 0.
\eQe
\hfill $\qed$

\end{solution}

\pagebreak 

\begin{question}[4]
\end{question}
\begin{solution}
\end{solution}
\bigskip

\begin{question}[5]
\end{question}
\begin{solution}

\end{solution}

\bigskip

\begin{question}[6]
\end{question}
\begin{solution}
We are given that $\det(A) = \det(A^T)$. Let $\lambda$ be an eigenvalue
of $A$. It follows that $\det(A - \lambda I ) = 0$. Since
$\det(A) = \det(A^T)$, we have 
\eQb
\det(A -\lambda I) &=& \det((A - \lambda I)^T) \\
&=& \det(A^T - \lambda I). \\
\eQe 
Hence, $\det(A^T - \lambda I) = 0$ holds as well.
Therefore, $\lambda$ is an eigenvalue of $A^T$.
Since $\lambda$ was an arbitrary eigenvalue of $A$, we have shown that 
any eigenvalue of $\lambda$ of $A$ is also an eigenvalue of $A^T$.
\hfill $\qed$ 
\end{solution}

\end{document}
