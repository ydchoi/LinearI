\documentclass{article} % For LaTeX2e
\usepackage{nips14submit_e,times}
\usepackage{amsmath}
\usepackage{amsthm}
\usepackage{amssymb}
\usepackage{mathtools}
\usepackage{hyperref}
\usepackage{url}
\usepackage{algorithm}
\usepackage[noend]{algpseudocode}
%\documentstyle[nips14submit_09,times,art10]{article} % For LaTeX 2.09

\usepackage{graphicx}
\usepackage{caption}
\usepackage{subcaption}

\def\eQb#1\eQe{\begin{eqnarray*}#1\end{eqnarray*}}
\def\eQnb#1\eQne{\begin{eqnarray}#1\end{eqnarray}}
\providecommand{\e}[1]{\ensuremath{\times 10^{#1}}}
\providecommand{\pb}[0]{\pagebreak}

\newcommand{\E}{\mathrm{E}}
\newcommand{\Var}{\mathrm{Var}}
\newcommand{\Cov}{\mathrm{Cov}}

\def\Qb#1\Qe{\begin{question}#1\end{question}}
\def\Sb#1\Se{\begin{solution}#1\end{solution}}

\newenvironment{claim}[1]{\par\noindent\underline{Claim:}\space#1}{}
\newtheoremstyle{quest}{\topsep}{\topsep}{}{}{\bfseries}{}{ }{\thmname{#1}\thmnote{ #3}.}
\theoremstyle{quest}
\newtheorem*{definition}{Definition}
\newtheorem*{theorem}{Theorem}
\newtheorem*{lemma}{Lemma}
\newtheorem*{question}{Question}
\newtheorem*{preposition}{Preposition}
\newtheorem*{exercise}{Exercise}
\newtheorem*{challengeproblem}{Challenge Problem}
\newtheorem*{solution}{Solution}
\newtheorem*{remark}{Remark}
\usepackage{verbatimbox}
\usepackage{listings}
\title{Linear Algebra I: \\
Problem Set II}


\author{
Youngduck Choi \\
CIMS \\
New York University\\
\texttt{yc1104@nyu.edu} \\
}


% The \author macro works with any number of authors. There are two commands
% used to separate the names and addresses of multiple authors: \And and \AND.
%
% Using \And between authors leaves it to \LaTeX{} to determine where to break
% the lines. Using \AND forces a linebreak at that point. So, if \LaTeX{}
% puts 3 of 4 authors names on the first line, and the last on the second
% line, try using \AND instead of \And before the third author name.

\newcommand{\fix}{\marginpar{FIX}}
\newcommand{\new}{\marginpar{NEW}}

\nipsfinalcopy % Uncomment for camera-ready version

\begin{document}


\maketitle

\begin{abstract}
This work contains the solutions to the problem set II
of Linear Algebra I 2015 at Courant Institute of Mathematical Sciences.
\end{abstract}

\bigskip

\begin{question}[1]
\end{question}
\begin{solution}
By the definition of the map $O$,
and the fact that $0x = 0$ for all $x \in X$, as $X$ is a linear space,
we have  
\eQb
O(ax) &=& 0 \\
&=& a 0 \\
&=& a O(x), \\
\eQe
for all $x \in X$.
Furthermore, again by the definition of the map $O$, it follows that
\eQb
O(x+y) &=& 0 \\
&=& 0 + 0 \\
&=& O(x) + O(y),
\eQe
for all $x,y \in X$.
Therefore, we have shown that $O$ is linear. 

\smallskip
Let $A$ be another map such that $A \circ A  = O$. Note that this implicitly 
gives that the linear space $X$ under consideration is a non-trivial one, as
otherwise the only map available is $O$.  
Suppose for sake of contradiction that there exists an inverse map 
of $A$, denoted by $A^{-1}$. Note that by the existence of an inverse map,
$A$ and $A^{-1}$ are bijective maps, and thus the domain and range of these
maps are all $X$. Observe that $O \circ A^{-1} = O$. 
By using the associativity of map composition,
$O = A \circ A$ and $O = O \circ A^{-1}$, we have
\eQb
I &=& A \circ A^{-1} \\
&=& A \circ I \circ A^{-1} \\ 
&=& (A \circ A) \circ A^{-1} \circ A^{-1} \\
&=& (O \circ A^{-1}) \circ A^{-1} \\
&=& O \circ A^{-1} \\
&=& O. 
\eQe
We have reached a conclusion that $I = O$. 
As discussed before, the linear space $X$ is nontrivial. 
Hence, $I = O$ is a contradiction. Consequently,
$A$ does not have an inverse map.
\hfill $\qed$ 

\end{solution}

\pagebreak

\begin{question}[2]
\end{question}
\begin{solution}
Let $x \in X$. Consider $[A,[B,C]] + [C,[A,B]] + B,[C,A]](x)$. Given the
definition of the commutator, it follows that
\eQb
[A,[B,C]](x) &=& [A, B\circ C - C\circ B](x) \\
&=& (A \circ (B\circ C - C \circ B) - (B \circ C - C \circ B) \circ A)(x) \\
&=& (A \circ B \circ C - A \circ C \circ B - B \circ C \circ A +
C \circ B \circ A)(x). \\
\eQe
By the associativity of map composition and the linearity of the maps, which
is called "Composition is distributive with respect to the addition of
linear maps" in Lax, 
it follows that
\eQb
[A[B,C]](x) &=& 
(A \circ B \circ C)(x) - (A \circ C \circ B)(x) - (B \circ C \circ A)(x)
+ (C \circ B \circ A)(x) \\
&=& A(B(C(x))) - A(C(B(x))) - B(C(A(x))) + C(B(A(x))).
\eQe
By symmetry, it follows that
\eQb
[C,[A,B]](x) &=& C(A(B(x))) - C(B(A(x))) - A(B(C(x))) + B(A(C(x))), \\ 
\eQe
and 
\eQb
[B,[C,A]](x) &=& B(C(A(x))) - B(A(C(x))) - C(A(B(x))) + A(C(B(x))) \\.
\eQe
It follows that 
\eQb
[A,[B,C]] + [C,[A,B]] + [B,[C,A]](x) &=& 0.
\eQe
As $x$ was arbitrary, we have shown that $[A,[B,C]] + [C,[A,B]] + [B,[C,A]] =
O$. 
\hfill $\qed$
\end{solution}
\bigskip

\begin{question}[3]
\end{question}
\begin{solution}
Let $T$ be a linear map, and $T^{'}$ be a transpose map of $T$. Assume
that $T$ and $T^{'}$ are both invertible. The statement that we want
to show asserts that
\eQb
(T^{-1})^{'} &=& (T^{'})^{-1},
\eQe 
which states that $(T^{-1})^{'}$ is the inverse of $T^{'}$. By the definition
of inverse, it suffices to show that
\eQb
T^{'}(T^{-1})^{'} &=& I^{'}, \\
(T^{-1})^{'}T^{'} &=& I^{'}. \\
\eQe
We have previously shown in class that for any linear map $(ST)^{'} = 
T^{'}S^{'}$. It follows that 
\eQb
T^{'}(T^{-1})^{'} &=& (T^{-1}T)^{'} \\ 
&=& I^{'}, \\
(T^{-1})^{'}T^{'} &=& (T T^{-1})^{'} \\ 
&=& I^{'}. \\
\eQe 
\hfill $\qed$
\end{solution}

\pagebreak 

\begin{question}[4]
\end{question}
\begin{solution}
We have that the dimension of the range space of $T$ is $1$. Let $r$ 
be the vector that spans the range space. Extend the set $\{ r\}$ 
with linearly independent vectors to obtain the basis set that spans $X$ 
entire space. Denote this set as $\{ r, v_1, ..., v_{n-1} \}$. Observe
that $T(v_i) = 0$ for all $i$ as they belong to the null-space. Let
$x \in X$. Then, there exists a set of scalars such that
\eQb
x &=& a_1 r + a_2 v_1 ... + a_n v_{n-1} .
\eQe
Consider $T(x)$. As $\{r\}$ spans the range space, there exists a scalar
$c$ such that $T(x) = cr$. 
It follows that
\eQb
tr &=& T(a_1 r + a_2 v_1 ... + a_n v_{n-1}), \\
\eQe 
which by linearity of $T$ and the property of null-space mentioned above,
can be simplified as,
\eQb
cr &=& rT(a_1) + a_2T(v_1) ... + a_n T(v_{n-1}), \\
&=& rT(a_1). \\
\eQe
Hence, we have $c = T(a_1)$. It follows that
\eQb
T^2(x) &=& T(T(x)) \\
&=& T(cr) \\
&=& cT(r) \\
&=& cT(x). \\
\eQe 

We have shown that $T^2 = cT$. Now, assume that $c \neq 1$. 
Consider the linear map $I + \dfrac{1}{1-c}T$. By the linearity of 
maps, which allows the compositions to distribute, it follows that
\eQb
(I - T) \circ (I + \dfrac{1}{1-c}T) &=& I \circ I 
+ I \circ \dfrac{1}{1-c}T - T\circ I - T \circ \dfrac{1}{1-c}T \\
&=& I + \dfrac{1}{1-c}T - T - \dfrac{1}{1-c}T^2 \\
&=& I + \dfrac{1}{1-c}T - T - \dfrac{c}{1-c}T \\
&=& I. \\ 
\eQe
Hence, we have shown that if $c \neq 1$, $I - T$ is invertible.
\hfill $\qed$ 
\end{solution}
\bigskip

\begin{question}[5]
\end{question}
\begin{solution}
Let $A$ and $B$ be a $2 \times 2$ matrix such that
\eQb
A &=& \begin{bmatrix}
1 & 1 \\
1 & 1 \\
\end{bmatrix} \\
B &=& \begin{bmatrix}
1 & 1 \\
-1 & -1 \\
\end{bmatrix}.
\\
\eQe 
Observe that both $A$ and $B$ are nonzero matrices and  
\eQb
AB &=& \begin{bmatrix}
1 & 1 \\
1 & 1 \\
\end{bmatrix}
\begin{bmatrix}
1 & 1 \\
-1 & -1 \\
\end{bmatrix} = 
\begin{bmatrix}
0 & 0 \\
0 & 0 \\
\end{bmatrix}.
\eQe
\hfill $\qed$
\end{solution}

\bigskip

\end{document}
